\documentclass[11pt,a4paper, oneside, oldfontcommands]{memoir}
\usepackage[utf8]{inputenc}
\usepackage[T1]{fontenc}
\usepackage{microtype}
\usepackage[dvips]{graphicx}
\usepackage{xcolor}
\usepackage{times}
\usepackage[most]{tcolorbox}
\usepackage{geometry}

\usepackage[
breaklinks=true,colorlinks=true,
%linkcolor=blue,urlcolor=blue,citecolor=blue,% PDF VIEW
linkcolor=black,urlcolor=black,citecolor=black,% PRINT
bookmarks=true,bookmarksopenlevel=2]{hyperref}


\geometry{
  left=25mm,
  right=25mm,
  top=25mm,
  bottom=25mm,
  heightrounded,% better use it
}

%My nice quotation style
\newtcolorbox{siderules}[1][]{blanker, breakable, 
     left=6mm,right=0mm, top=1mm, bottom=1mm,
     borderline west={1pt}{0pt}{black},
     parbox=false, #1}



\OnehalfSpacing
%\linespread{1.3}

%%% CHAPTER'S STYLE
\chapterstyle{bianchi}
%\chapterstyle{ger}
%\chapterstyle{madsen}
%\chapterstyle{ell}
%%% STYLE OF SECTIONS, SUBSECTIONS, AND SUBSUBSECTIONS
\setsecheadstyle{\Large\bfseries\sffamily\raggedright}
\setsubsecheadstyle{\large\bfseries\sffamily\raggedright}
\setsubsubsecheadstyle{\bfseries\sffamily\raggedright}


%%% STYLE OF PAGES NUMBERING
%\pagestyle{companion}\nouppercaseheads 
%\pagestyle{headings}
%\pagestyle{Ruled}
\pagestyle{plain}
\makepagestyle{plain}
\makeevenfoot{plain}{\thepage}{}{}
\makeoddfoot{plain}{}{}{\thepage}
\makeevenhead{plain}{}{}{}
\makeoddhead{plain}{}{}{}

%Define G
\newcommand{\G}{\textbf{\textit{G}}}
%Define V
\newcommand{\V}{\textbf{\textit{V}}}
%Define E
\newcommand{\E}{\textbf{\textit{E}}}

\maxsecnumdepth{subsection} % chapters, sections, and subsections are numbered
\maxtocdepth{subsection} % chapters, sections, and subsections are in the Table of Contents

% Le bloc qui suit permet de ne pas afficher les numéros devants les sections
\makeatletter
\def\@seccntformat#1{%
  \expandafter\ifx\csname c@#1\endcsname\c@section\else
  \csname the#1\endcsname\quad
  \fi}
\makeatother

%Ceci devrait retirer le 'Chapter N'
\makeatletter
\def\@makechapterhead#1{%
  \vspace*{50\p@}%
  {\parindent \z@ \raggedright \normalfont
    \interlinepenalty\@M
    \Large \bfseries #1\par\nobreak
    \vskip 40\p@
  }}
\def\@makeschapterhead#1{%
  \vspace*{50\p@}%
  {\parindent \z@ \raggedright
    \normalfont
    \interlinepenalty\@M
    \Large \bfseries  #1\par\nobreak
    \vskip 40\p@
  }}
\makeatother

\begin{document}

\thispagestyle{empty}

{%%%
\sffamily
\centering
\Large

~\vspace{\fill}

{\huge 
\textbf{Devoir 3}
}

\vspace{2.5cm}

{
BLAIS REGOUT, Lucien\\[1em]
SAVOIE, Olivier
}

\vspace{3.5cm}

IFT436 - Algoritmes et structures de données\\[1em]

Faculté des Sciences\\
Université de Sherbrooke

\vspace{3.5cm}

Présenté au/à?? Prof. BLONDIN, Michael

\vspace{\fill}

Jeudi, 10 octobre 2019

%%%
}%%%
\chapter{Première Question}
\section{a)}
  \textbf{\textit{V}} = Les bassins du manège aquatique (sommets).
  \textbf{\textit{E}} = Les corridors entre les bassins, potentiellement des glissades(arêtes).
  \textbf{\textit{G}} = Manège aquatique, est dirigé, du à l'élevation.
  Élévation dénoté de l'énoncé suivant:
  \begin{siderules}
    Puisque les bassins sont situés de plus en plus bas le long d'une colline, si on peut atteindre un bassin j à partir d'un bassin i, alors on ne peut pas atteindre le bassin i à partir du bassin j.
  \end{siderules}

\section{b)}
  \textbf{\textit{G}} ne peut pas être un cycle dû contraintes fournis ci-dessous. Grâce à cette première contraites plus bas, il est impossible qu'un cycle soit établi avec un seul bassin considérant qu'il y a aucun corridor qui part d'un dit bassin en arrivant dans ce même bassin. Finalement, la seconde contrainte implique que dû à l'inclinaison, l'un ne peut pas revenir au bassin précédent. Cela implique qu'il y a aucun cycle possible et que de ce fait, le manège aquatique \G{} est acyclique et dirigé.
  \begin{siderules}
    Il n'y a \textit{pas} de corridor passant d'un bassin vers lui-même.
  \end{siderules}
  \begin{siderules}
    Puisque les bassins sont situés de plus en plus bas, le long d'une colline, si on peut atteindre un bassin \textit{j} à partir d'un bassin \textit{i}, alors on ne peut pas atteindre le bassin \textit{i} à partir du bassin \textit{j}.
  \end{siderules}

\subsection{First subsection}

Lorem ipsum dolor sit amet, consectetur adipiscing elit, sed do eiusmod tempor incididunt ut labore et dolore magna aliqua. Ut enim ad minim veniam, quis nostrud exercitation ullamco laboris nisi ut aliquip ex ea commodo consequat. Duis aute irure dolor in reprehenderit in voluptate velit esse cillum dolore eu fugiat nulla pariatur. Excepteur sint occaecat cupidatat non proident, sunt in culpa qui officia deserunt mollit anim id est laborum.

\subsection{Second subsection}

Lorem ipsum dolor sit amet, consectetur adipiscing elit, sed do eiusmod tempor incididunt ut labore et dolore magna aliqua. Ut enim ad minim veniam, quis nostrud exercitation ullamco laboris nisi ut aliquip ex ea commodo consequat. Duis aute irure dolor in reprehenderit in voluptate velit esse cillum dolore eu fugiat nulla pariatur. Excepteur sint occaecat cupidatat non proident, sunt in culpa qui officia deserunt mollit anim id est laborum.

\section{Third section}
Lorem ipsum dolor sit amet, consectetur adipiscing elit, sed do eiusmod tempor incididunt ut labore et dolore magna aliqua. Ut enim ad minim veniam, quis nostrud exercitation ullamco laboris nisi ut aliquip ex ea commodo consequat. Duis aute irure dolor in reprehenderit in voluptate velit esse cillum dolore eu fugiat nulla pariatur. Excepteur sint occaecat cupidatat non proident, sunt in culpa qui officia deserunt mollit anim id est laborum.

Lorem ipsum dolor sit amet, consectetur adipiscing elit, sed do eiusmod tempor incididunt ut labore et dolore magna aliqua. Ut enim ad minim veniam, quis nostrud exercitation ullamco laboris nisi ut aliquip ex ea commodo consequat. Duis aute irure dolor in reprehenderit in voluptate velit esse cillum dolore eu fugiat nulla pariatur. Excepteur sint occaecat cupidatat non proident, sunt in culpa qui officia deserunt mollit anim id est laborum.

\section{Fourth section}


\end{document}