\documentclass[11pt,a4paper, oneside, oldfontcommands]{memoir}
\usepackage[french]{babel}
\selectlanguage{french}
\usepackage[utf8]{inputenc}
\usepackage[T1]{fontenc}
\usepackage{microtype}
\usepackage[dvips]{graphicx}
\usepackage{xcolor}
\usepackage{times}
\usepackage[most]{tcolorbox}
\usepackage{geometry}
\usepackage{amsmath}
\usepackage[french, onelanguage, noend, ruled, vlined]{algorithm2e}
\usepackage{tikz-qtree}

\usepackage[
breaklinks=true,colorlinks=true,
%linkcolor=blue,urlcolor=blue,citecolor=blue,% PDF VIEW
linkcolor=black,urlcolor=black,citecolor=black,% PRINT
bookmarks=true,bookmarksopenlevel=2]{hyperref}


\geometry{
  left=25mm,
  right=25mm,
  top=25mm,
  bottom=25mm,
  heightrounded,% better use it
}

%My nice quotation style
\newtcolorbox{siderules}[1][]{blanker, breakable, 
     left=6mm,right=0mm, top=1mm, bottom=1mm,
     borderline west={1pt}{0pt}{black},
     parbox=false, #1}



\OnehalfSpacing
%\linespread{1.3}

%%% CHAPTER'S STYLE
\chapterstyle{bianchi}
%\chapterstyle{ger}
%\chapterstyle{madsen}
%\chapterstyle{ell}
%%% STYLE OF SECTIONS, SUBSECTIONS, AND SUBSUBSECTIONS
\setsecheadstyle{\Large\bfseries\sffamily\raggedright}
\setsubsecheadstyle{\large\bfseries\sffamily\raggedright}
\setsubsubsecheadstyle{\bfseries\sffamily\raggedright}


%%% STYLE OF PAGES NUMBERING
%\pagestyle{companion}\nouppercaseheads 
%\pagestyle{headings}
%\pagestyle{Ruled}
\pagestyle{plain}
\makepagestyle{plain}
\makeevenfoot{plain}{\thepage}{}{}
\makeoddfoot{plain}{}{}{\thepage}
\makeevenhead{plain}{}{}{}
\makeoddhead{plain}{}{}{}

%Define G
\newcommand{\G}{$\textbf{\textit{G}}$}
%Define V
\newcommand{\V}{$\textbf{\textit{V}}$}
%Define E
\newcommand{\E}{$\textbf{\textit{E}}$}

\maxsecnumdepth{subsection} % chapters, sections, and subsections are numbered
\maxtocdepth{subsection} % chapters, sections, and subsections are in the Table of Contents

% Le bloc qui suit permet de ne pas afficher les numéros devants les sections
\makeatletter
\def\@seccntformat#1{%
  \expandafter\ifx\csname c@#1\endcsname\c@section\else
  \csname the#1\endcsname\quad
  \fi}
\makeatother

%Ceci devrait retirer le 'Chapter N'
\makeatletter
\def\@makechapterhead#1{%
  \vspace*{50\p@}%
  {\parindent \z@ \raggedright \normalfont
    \interlinepenalty\@M
    \Large \bfseries #1\par\nobreak
    \vskip 40\p@
  }}
\def\@makeschapterhead#1{%
  \vspace*{50\p@}%
  {\parindent \z@ \raggedright
    \normalfont
    \interlinepenalty\@M
    \Large \bfseries  #1\par\nobreak
    \vskip 40\p@
  }}
\makeatother

\begin{document}

\thispagestyle{empty}

{%%%
\sffamily
\centering
\Large

~\vspace{\fill}

{\huge 
\textbf{Devoir 3}
}

\vspace{2.5cm}

{
BLAIS REGOUT, Lucien\\
SAVOIE, Olivier
}

\vspace{3.5cm}

IFT436 - Algoritmes et structures de données\\[1em]

Faculté des Sciences\\
Université de Sherbrooke

\vspace{3.5cm}

Présenté à Pr BLONDIN, Michael

\vspace{\fill}

Jeudi, 10 octobre 2019

%%%
}%%%
\chapter{Première Question}
\section{a)}
  \textbf{\textit{V}} = Les bassins du manège aquatique (sommets).
  \textbf{\textit{E}} = Les corridors entre les bassins, potentiellement des glissades(arêtes).
  \textbf{\textit{G}} = Manège aquatique, est dirigé, du à l'élevation.
  Élévation dénoté de l'énoncé suivant:
  \begin{siderules}
    Puisque les bassins sont situés de plus en plus bas, le long d'une colline, si on peut atteindre un bassin $j$ à partir d'un bassin $i$, alors on ne peut pas atteindre le bassin $i$ à partir du bassin $j$.
  \end{siderules}


\section{b)}
  \textbf{\textit{G}} ne peut pas être un cycle dû contraintes fournis ci-dessous. Grâce à cette première contraites plus bas, il est impossible qu'un cycle soit établi avec un seul bassin considérant qu'il y a aucun corridor qui part d'un dit bassin en arrivant dans ce même bassin. Finalement, la seconde contrainte implique que dû à l'inclinaison, l'un ne peut pas revenir au bassin précédent. Cela implique qu'il y a aucun cycle possible et que de ce fait, le manège aquatique \G{} est acyclique et dirigé.
  \begin{siderules}
    Il n'y a \textit{pas} de corridor passant d'un bassin vers lui-même.
  \end{siderules}
  \begin{siderules}
    Puisque les bassins sont situés de plus en plus bas, le long d'une colline, si on peut atteindre un bassin $j$ à partir d'un bassin $i$, alors on ne peut pas atteindre le bassin $i$ à partir du bassin $j$.
  \end{siderules}


\section{c)}
  Si le graphe \G{} est considéré non dirigé, il y a plusieurs arrangements d'arêtes qui rendrait \G{} cyclique. Toutefois, il y a toujours la possibilité qu'aucun cycle, simple ou non, ne soit présent et que le nombre d'arête soit équivalent au nombre de sommet $- 1$. De sorte que les propriétés suivantes soit respectées.
  \begin{itemize}
    \item \G{} est \textit{connexe}
    \item \G{} est \textit{acyclique}
    \item $|\E{}|=|\V{}|$ \textbf{YALL C'EST PAS BON!!!!!}
  \end{itemize}

\section{d)}
  \begin{gather*}
    \forall n \geq 1\quad\\
    min = n-1\quad\\
    max = \sum_{i=1}^{n} {n-i}=\sum_{i=1}^{n} {n}-\sum_{i=1}^{n} {i}=n^{2}-\frac{n}{2}(n+1)=\frac{n^{2}-n}{2}
  \end{gather*}

\section{e)}
  \textbf{Bassin départ :} De toutes les paires de sommets composants les arêtes, le sommet qui se retrouvera uniquement en index $0$ de la paire, à la gauche, qui est donc toujours le sommet de départ, sera donc considéré comme sommet de départ.\\[1em]
  \textbf{Bassin d'arrivée :} Suivre le même résonnement, mais pour le sommet qui se retrouve toujours en index $1$ de la paire et qui est donc toujours le bassin d'arrivée dans toutes les paires, sera donc considéré comme sommet d'arrivée.


  % package algorithm2e:
  % KwIn, KwOut, KwData, KwResult
  % KwTo, KwFrom
  % KwRet, Return
  % Begin
  % Repeat
  % If ElseIf, Else
  % Switch, Case, Other
  % For, ForPar, ForEach, ForAll, While
\section{f)}
  \IncMargin{1em}
  \begin{algorithm}[H]

    \caption{Calcul de la séquence qui possède un temps maximal passé dans une attraction \G{}}
    \DontPrintSemicolon
    \LinesNumbered
    \SetAlgoLined
    \SetKwProg{Def}{}{:}{}
    \SetKwFunction{Parcours}{parcours}

    \KwIn{Manège $\G{}$ tel que $\G{} = (\V{},\E{})$ et sommet initial $u$ tel que $u$ appartien à $\V{}$.}
    \KwResult{Une séquence $S$ de bassins de temps maximal dans le manège \G{}.}
    $S \leftarrow [ ]$\;
    $Ttot \leftarrow 0$\;
    $Parent \leftarrow u$\;

    \Def{\Parcours($x$)}{
      $SPrime \leftarrow S$\;
      $Temps \leftarrow Ttot$\;

      \If{$x$ non marquée}{
        $Temps \leftarrow Temps + c[Parent, x]$\;
        \textbf{marquer} $x$\;
        \textbf{ajouter} $x$ à $SPrime$\;
        
        \If{$Temps > Ttot$}{
          $Ttot \leftarrow Temps$\;
          $S \leftarrow SPrime$\;
        }

        \For{$u \in \V{} : x \rightarrow y$}{
          $Parent \leftarrow x$\;
          \Parcours{$y$}
        }
      }
    }
    \Parcours{$u$}
    \Return{$S$}
  \end{algorithm}
  \DecMargin{1em}

\section{g)}
  \IncMargin{1em}
  \begin{algorithm}[H]

    \caption{Calcul de la séquence qui possède un temps maximal passé dans une attraction \G{}, donné $5$ remontées}
    \DontPrintSemicolon
    \LinesNumbered
    \SetAlgoLined
    \SetKwProg{Def}{}{:}{}
    \SetKwFunction{Parcours}{parcours}

    \KwIn{Manège \G{} tel que $\G{} = (\V{},\E{})$, que le sommet initial $u$ tel que $u \in \V{}$, que $v \in \V{}$, ainsi que $v$ est le bassin finale.}
    \KwResult{Une séquence $S$ de bassins, de temps maximal dans le manège \G{}, considérant que l'utilisateur peut remonter au bassin initial un maximum de $5$ fois.}
    $S \leftarrow [\ ]$\;
    $Ttot \leftarrow 0$\;
    $Parent \leftarrow u$\;
    $Ascension \leftarrow 5$\;

    \Def{\Parcours($x$)}{
      $SPrime \leftarrow S$\;
      $Temps \leftarrow Ttot$\;

      \If{$x$ non marquée}{
        $Temps \leftarrow Temps + c[Parent, x]$\;
        \textbf{marquer} $x$\;
        \textbf{ajouter} $x$ à $SPrime$\;
        
        \If{$Temps > Ttot$}{
          $Ttot \leftarrow Temps$\;
          $S \leftarrow SPrime$\;
        }

        \If{prochain voisin de $x = v$}{
          $Ascension \leftarrow Acension - 1$\;
          \textbf{retirer} les marqueurs\;
          $Parent \leftarrow S[0]$\;
          \Parcours{$Parent$}
        }

        \Else{
          \For{$u \in \V{} : x \rightarrow y$}{
            $Parent \leftarrow x$\;
            \Parcours{$y$}
          }
        }
      }
    }
    \Parcours{$u$}
    \Return{$S$}
  \end{algorithm}
  \DecMargin{1em}

  \BlankLine

  Ici, il fera exactement comme l'algorithme précédent, à l'exception que lorsqu'il se rendra au plus long parcours précédent le bassin final,
  et se 5 fois. La 5e fois, le sera possiblement différent mais restera cependant couvers par l'algorithme.\\[1em]

\chapter{Seconde Question}
\section{a)}

\IncMargin{1em}
\begin{algorithm}[H]

  \caption{Tri d'une séquence et calcul des valeurs modales}
  \DontPrintSemicolon
  \LinesNumbered
  \SetAlgoLined
  \SetKwFunction{TMode}{trouverMode}
  \SetKwFunction{Med}{mediane}
  \SetKwProg{Def}{}{:}{}

  \KwIn{Une séquence $S$ non vide d'éléments comparables.}
  \KwResult{Les valeurs modales$m$ trouvées dans la séquence $S$}
  \BlankLine
  $mode \leftarrow [\ ]$\;
  $modeQty \leftarrow 0$\;

  \Def{\TMode($x$)}{

    \If{$|x|=0$}{
      \Return{$mode$}
      }
    \Else{
      $pivot \leftarrow$ \Med{$x$}\;
      $gauche \leftarrow [x \in S: x < pivot]$\;
      $centre \leftarrow [x \in S: x = pivot]$\;
      $droite \leftarrow [x \in S: x > pivot]$\;
      
      \If{$|centre|=modeQty$}{
        \textbf{ajouter} $pivot$ à $mode$\;
      }
      
      \If{$|centre|>modeQty$}{
        $modeQty \leftarrow |centre|$\;
        $mode \leftarrow [pivot]$
      }

      \Else{
        \TMode{$gauche$}\;
        \TMode{$droite$}
      }
    }
  }
  \Return{\TMode{$S$}}  
\end{algorithm}
\DecMargin{1em}


\section{b)}

\resizebox{0.9\textwidth}{!}{
  \begin{tikzpicture}
\Tree
[.{[10, 20, 20, 20, 10, 30, 70, 20, 80, 30, 60, 40, 20, 10, 50, 10, 70, 80]}
  [.{[10, 10, 10, 10]}
    {[ ]}
    {[10, 10, 10, 10]}
    {[ ]}
  ]
  {[20, 20, 20, 20, 20]}
  [.{[30, 70, 80, 30, 60, 40, 50, 70, 80]}
    [.{[30, 30, 40, 50]}
      {[ ]}
      {[30, 30]}
      [.{[40, 50]}
        {[ ]}
        {[40]}
        [.{[50]}
          {[ ]}
          {[50]}
          {[ ]}
        ]
      ]
    ]
    {[60]}
    [.{[70, 80, 70, 80]}
      {[ ]}
      {[70, 70]}
      [.{[80,80]}
        {[ ]}
        {[80, 80]}
        {[ ]}
      ]
    ]
  ]
]
  \end{tikzpicture}
}
\end{document}