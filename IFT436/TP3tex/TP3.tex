\documentclass[11pt,a4paper, oneside, oldfontcommands]{memoir}
\usepackage[french]{babel}
\selectlanguage{french}
\usepackage[utf8]{inputenc}
\usepackage[T1]{fontenc}
\usepackage{microtype}
\usepackage[dvips]{graphicx}
\usepackage{xcolor}
\usepackage{times}
\usepackage[most]{tcolorbox}
\usepackage{geometry}
\usepackage{amsmath}
\usepackage{amssymb}
\usepackage[french, onelanguage, noend, ruled, vlined]{algorithm2e}
\usepackage{tikz-qtree}
\usepackage{fontawesome}
\usepackage{tikz}



\usetikzlibrary{shapes,arrows,positioning,decorations.pathreplacing,calc, automata}

\DeclareMathAlphabet\mathbfcal{OMS}{cmsy}{b}{n}
%usage : \mathcal{} et \mathbfcal{} for bold

\usepackage[
breaklinks=true,colorlinks=true,
linkcolor=black,urlcolor=black,citecolor=black,
bookmarks=true,bookmarksopenlevel=2]{hyperref}


\newcommand\mycommfont[1]{\footnotesize\ttfamily\textcolor{blue}{#1}}
\SetCommentSty{mycommfont}

\geometry{
  left=25mm,
  right=25mm,
  top=25mm,
  bottom=25mm,
  heightrounded,% better use it
}

%My nice quotation style
\newtcolorbox{siderules}[1][]{blanker, breakable, 
     left=6mm,right=0mm, top=1mm, bottom=1mm,
     borderline west={1pt}{0pt}{black},
     parbox=false, #1}



\OnehalfSpacing
%\linespread{1.3}

%%% CHAPTER'S STYLE
\chapterstyle{bianchi}
%\chapterstyle{ger}
%\chapterstyle{madsen}
%\chapterstyle{ell}
%%% STYLE OF SECTIONS, SUBSECTIONS, AND SUBSUBSECTIONS
\setsecheadstyle{\Large\bfseries\sffamily\raggedright}
\setsubsecheadstyle{\large\bfseries\sffamily\raggedright}
\setsubsubsecheadstyle{\bfseries\sffamily\raggedright}


%%% STYLE OF PAGES NUMBERING
%\pagestyle{companion}\nouppercaseheads 
%\pagestyle{headings}
%\pagestyle{Ruled}
\pagestyle{plain}
\makepagestyle{plain}
\makeevenfoot{plain}{\thepage}{}{}
\makeoddfoot{plain}{}{}{\thepage}
\makeevenhead{plain}{}{}{}
\makeoddhead{plain}{}{}{}

%Define G
\newcommand{\G}{$\mathbfcal{G}$}
%Define V
\newcommand{\V}{$\textit{V}$}
%Define E
\newcommand{\E}{$\textit{E}$}

\maxsecnumdepth{subsection} % chapters, sections, and subsections are numbered
\maxtocdepth{subsection} % chapters, sections, and subsections are in the Table of Contents

% Le bloc qui suit permet de ne pas afficher les numéros devants les sections
\makeatletter
\def\@seccntformat#1{%
  \expandafter\ifx\csname c@#1\endcsname\c@section\else
  \csname the#1\endcsname\quad
  \fi}
\makeatother

%Ceci devrait retirer le 'Chapter N'
\makeatletter
\def\@makechapterhead#1{%
  \vspace*{50\p@}%
  {\parindent \z@ \raggedright \normalfont
    \interlinepenalty\@M
    \Large \bfseries #1\par\nobreak
    \vskip 40\p@
  }}
\def\@makeschapterhead#1{%
  \vspace*{50\p@}%
  {\parindent \z@ \raggedright
    \normalfont
    \interlinepenalty\@M
    \Large \bfseries  #1\par\nobreak
    \vskip 40\p@
  }}
\makeatother

\begin{document}

\thispagestyle{empty}

{%%%
\sffamily
\centering
\Large

~\vspace{\fill}

{\huge 
\textbf{Devoir 3}
}

\vspace{2.5cm}

{
BLAIS REGOUT, Lucien - 18073291\\
SAVOIE, Olivier - 18114274
}

\vspace{3.5cm}

IFT436 - Algoritmes et structures de données\\[1em]

Faculté des Sciences\\
Université de Sherbrooke

\vspace{3.5cm}

Présenté à Pr BLONDIN, Michael

\vspace{\fill}

Jeudi, 10 octobre 2019

%%%
}%%%
\chapter{Question 1}
\section{a)}
  \noindent
  $\V{}$ = Les bassins du manège aquatique (sommets).\\[1em]
  \E{} = Les corridors entre les bassins, potentiellement des glissades (arêtes).\\[1em]
  \G{} = Manège aquatique est dirigé.


\section{b)}
  \G{} ne peut pas être un cycle dû aux contraintes stipulées. Cette contrainte, ci-dessous, en témoigne.\\

  \begin{siderules}
    Il n'y a \textit{pas} de corridor passant d'un bassin vers lui-même.
  \end{siderules}
  
  \noindent
  Il est impossible qu'un cycle soit établi avec un seul bassin considérant qu'il y a aucun corridor qui part d'un dit bassin en arrivant dans ce même bassin.\\[1em]
  
  
  Finalement, cette seconde contrainte, implique que, dû à l'inclinaison, l'un ne peut pas revenir au bassin précédent. Cela implique qu'il y a aucun cycle possible et que de ce fait, le manège aquatique \G{} est acyclique et dirigé.\\
  
  \begin{siderules}
    Puisque les bassins sont situés de plus en plus bas, le long d'une colline, si on peut atteindre un bassin $j$ à partir d'un bassin $i$, alors on ne peut pas atteindre le bassin $i$ à partir du bassin $j$.
  \end{siderules}


\section{c)}
Si le graphe \G{} est considéré non dirigé, il y a plusieurs arrangements d'arêtes qui rendrait \G{} cyclique et pour lesquels $|\E{}| > |\V{}| - 1$. Ainsi, dans la majorité des cas, \G{} n'est pas un arbre, car il ne respecte pas les propriétés essentielles pour en être un. Toutefois, il y a toujours la possibilité qu'aucun cycle, simple ou non, ne s'y retrouve et que le nombre d'arête soit équivalent au nombre de sommet $- 1$. De sorte que les propriétés suivantes soit respectées.\\
  \begin{itemize}
    \item\G{} est $connexe$ et $|\textit{E}| = |\textit{V}| - 1$
  \end{itemize} 

\section{d)}
  \begin{gather*}
    min = n-1,\quad \forall n \geq 1\quad
  \end{gather*}
  Dans le cas minimal, les arêtes placées représenterons un graphe \G{} qui sera forcément un arbre, respectant les contraintes, ainsi que l'explication élaborée à la lettre précédente, \textbf{c)}.\\
  \begin{gather*}
    max = \sum_{i=1}^{n} {n-i}=\sum_{i=1}^{n} {n}-\sum_{i=1}^{n} {i}=n^{2}-\frac{n}{2}(n+1)=\frac{n^{2}-n}{2}
  \end{gather*}
  Advennant que chacun des bassins présents se voient attribués un corridor vers chacun des bassins soudjacents, l'équation maximal reflètera la composition du chemin le plus long. 

\section{e)}
  Le \textbf{bassin de départ} est le seul bassin qui ne vas jamais se retrouver à droite dans une paire de sommet, donc qui ne sera jamais $j$ dans un couple $(i,j)$.\\[1em]
  Le \textbf{bassin d'arrivée :} est le seul bassin qui ne vas jamais se retrouver à gauche dans une paire de sommet, donc qui ne sera jamais $i$ dans un couple $(i,j)$.

\newpage
\section{f)}
  \IncMargin{1em}
  \begin{algorithm}[H]

    \caption{Calcul de la séquence au temps maximal passé dans une attraction \G{}}
    \DontPrintSemicolon
    \LinesNumbered
    \SetAlgoLined
    \SetKwProg{Def}{}{:}{}
    \SetKwFunction{Parcours}{parcours}

    \KwIn{Tableau asssociatif $S: [n] \times [n] \rightarrow \mathbb{N} > 0$.}
    \KwResult{Le temps maximal.}
    $MaxSommet \leftarrow [n]$\;
    $SommeTemps \leftarrow 0$\;
    $Enfant \leftarrow 0$\;

    \Def{\Parcours($x$)}{

      \If{$x$ non marquée}{

        \For{$v \in S[x]$}{ %\tcp*{comments}
          \Parcours{$v$}\\
          $SommeTemps\leftarrow MaxSommet[Enfant] + c[x,v]$\;
          \If{$SommeTemps > MaxSommet[x]$}{
            $MaxSommet[x] \leftarrow SommeTemps$\;
          }
          \textbf{marquer} $x$\;
        }
      }
      $Enfant \leftarrow x-1$\;
      }
    \Parcours{$u$}\;
    \Return{$MaxSommet[0]$}
  \end{algorithm}
  \DecMargin{1em}

  \BlankLine

  
Dans cette algorithme, un sommet est marqué uniquement lorsque tous ces voisins ont été marqués.

Le fait d'utiliser un tableau dans lequel se trouve le temps maximal pour chaque sommet,
il n'est plus nécéssaire de faire tous les chemins possibles. On a juste besoin d'arreter,
dès qu'on voit un sommet marqué et prendre la valeur qui lui est attribué.

Cet algorithme se rappele récursivement sur chacun des enfants soudjacents, jusqu'à ce qu'on 
rencontre un sommet qui a déjè été marqué.

Ainsi, l’on peut en déduire que son temps d’exécution sera, dans le pire cas, $\textbf{O}(\E{} + \V{})$ ce 
qui respecte la contrainte comme quoi l'algorithme doit être e de $\textbf{O}(n + m)$.

\newpage
\section{g)}
\IncMargin{1em}
\begin{algorithm}[H]

  \caption{Calcul de la séquence au temps maximal passé dans une attraction \G{}, donné 5 remontées aribitrairement choisies.}
  \DontPrintSemicolon
  \LinesNumbered
  \SetAlgoLined
  \SetKwProg{Def}{}{:}{}
  \SetKwFunction{Parcours}{parcours}

  \KwIn{Liste d'adjacence, tableau asssociatif $c: [n] \times [n] -> N>0$ et tableau asssociatif $r: [n] -> \mathbb{N} > 0$}
  \KwResult{Le temps maximal.}
  $MaxSommet \leftarrow [n]$\;
  $SommeTemps \leftarrow 0$\;
  $Enfant \leftarrow 0$\;

  \Def{\Parcours($x$)}{

    \If{$x$ non marquée}{

      \For{$v \in S[x]$}{ %\tcp*{comments}
        \Parcours{$v$}\\
        $SommeTemps\leftarrow MaxSommet[Enfant] + c[x,v]$\;
        $MaxSommet[x-1] \leftarrow MaxSommet[x-1] + SommeTemps$\;
      }
      $MaxSommet[x-1] \leftarrow MaxSommet[x-1] + 5 \cdot r[x]$\;
    }
    $Enfant \leftarrow x-1$\;
    }
  \Parcours{$u$}\;
  \Return{$MaxSommet[0]$}
\end{algorithm}
\DecMargin{1em}

  \BlankLine

  Ici, il fera exactement comme l'algorithme précédent, à l'exception que lorsqu'il se rendra au plus long parcours précédent le bassin final,
  et se 5 fois. La 5e fois, le sera possiblement différent mais restera cependant couvers par l'algorithme.\\[1em]

\chapter{Question 2}
\section{a)}

\IncMargin{1em}
\begin{algorithm}[H]

  \caption{Tri d'une séquence et calcul des valeurs modales}
  \DontPrintSemicolon
  \LinesNumbered
  \SetAlgoLined
  \SetKwFunction{TMode}{trouverMode}
  \SetKwFunction{Med}{mediane}
  \SetKwProg{Def}{}{:}{}

  \KwIn{Une séquence $S$ non vide d'éléments comparables.}
  \KwResult{Les valeurs modales$m$ trouvées dans la séquence $S$}
  \BlankLine
  $mode \leftarrow [\ ]$\;
  $modeQty \leftarrow 0$\;

  \Def{\TMode($x$)}{

    \If{$|x|=0$}{
      \Return{$mode$}
      }
    \Else{
      $pivot \leftarrow$ \Med{$x$}\;
      $gauche \leftarrow [x \in S: x < pivot]$\;
      $centre \leftarrow [x \in S: x = pivot]$\;
      $droite \leftarrow [x \in S: x > pivot]$\;
      
      \If{$|centre|=modeQty$}{
        \textbf{ajouter} $pivot$ à $mode$\;
      }
      
      \If{$|centre|>modeQty$}{
        $modeQty \leftarrow |centre|$\;
        $mode \leftarrow [pivot]$
      }

      \Else{
        \TMode{$gauche$}\;
        \TMode{$droite$}
      }
    }
  }
  \Return{\TMode{$S$}}  
\end{algorithm}
\DecMargin{1em}


\section{b)}

\resizebox{0.9\textwidth}{!}{
  \begin{tikzpicture}
\Tree
[.{[10, 20, 20, 20, 10, 30, 70, 20, 80, 30, 60, 40, 20, 10, 50, 10, 70, 80]}
  [.{[10, 10, 10, 10]}
    {[ ]}
    {[10, 10, 10, 10]}
    {[ ]}
  ]
  {[20, 20, 20, 20, 20]}
  [.{[30, 70, 80, 30, 60, 40, 50, 70, 80]}
    [.{[30, 30, 40, 50]}
      {[ ]}
      {[30, 30]}
      [.{[40, 50]}
        {[ ]}
        {[40]}
        [.{[50]}
          {[ ]}
          {[50]}
          {[ ]}
        ]
      ]
    ]
    {[60]}
    [.{[70, 80, 70, 80]}
      {[ ]}
      {[70, 70]}
      [.{[80,80]}
        {[ ]}
        {[80, 80]}
        {[ ]}
      ]
    ]
  ]
]
  \end{tikzpicture}
}

\chapter{Question 3}
\section{a)}

\noindent
\begin{siderules}
  \textbf{Donnez un algorithme qui détermine si un graphe non dirigé est un arbre.}
\end{siderules}
\BlankLine

\IncMargin{1em}
\begin{algorithm}[H]

  \caption{Détermination de la propriété d'arbre d'un graphe.}
  \DontPrintSemicolon
  \LinesNumbered
  \SetAlgoLined
  \SetKwFunction{Vis}{visiter}
  \SetKwProg{Def}{}{:}{}

  \KwIn{Un graph \G{} tel que \G{}$=(\V{},\E{})$ est non dirié et un sommet $u \in \V{}$}
  \KwResult{Si le graphe est un arbre, ou non.}
  \BlankLine
  $Visites \leftarrow 0$\;

  \Def{\Vis{$x$}}{

    \If{$x$ est marqué}{
      \Return{$Non\_Arbre$}
    }
    \textbf{marquer} $x$\;
    $Visites \leftarrow Visites + 1$\;
    \ForAll{voisins de $x \in \V{} : x \rightarrow y$}{
      \Vis{$y$}\;
    }
  }
  \Vis{$u$}\;
  \If{$Visites \neq |\V{}|$}{
    \Return{$Non\_Arbre$}
  }
  \Else{
    \Return{$Est\_Arbre$}
  }
\end{algorithm}
\DecMargin{1em}

\BlankLine

Cet algorithme parcours le graph au complet. Si un sommet est visité plus d'une fois, c'est cyclique. Cela est uniquement véridicte lorsqu'on empêche de visiter le noeud précédents en visitant tous les voisins immédiats.\\

Deplus, tous les sommets peuvent être visiter moins de deux fois et tout de même il pourrait y manquer certains sommets, d'ou la vérification si le nombre de visites est bien égale a la somme des sommets dans le graphes!
\newpage
\section{b)}

\noindent
\begin{siderules}
  \textbf{Démontrez que le graphe complet \G{}$_n$ possède au moins $2^n$ arbres couvrants, $\forall n \in \mathbb{N} > 3$ .}
\end{siderules}

L'énoncé stipule que pour $n=4$ où n est le nombre de gites, il y a $2^4 = 16$ réseaux possibles. Si nous ajoutons une autre gites, nous obtenons ce dernier, à $5$ gites.

\begin{figure}[!h]
  \centering
%%%% 5 Pentagrame vide
\begin{tikzpicture}[node distance=2cm, baseline=0]
  %$ Sommets
  \tikzset{every state/.style={minimum size=10pt, inner sep=0pt, white}}

  \node[state] at (0, 0.3) (0) {};
  \node[state] at (6, 0.3) (1) {};
  \node[state] at (1,-3)   (2) {};
  \node[state] at (5,-3)   (3) {};
  \node[state] at (3, 2.5) (4) {};

  %% Arêtes
  \path[-, gray]
  (0) edge node {} (1)
  (0) edge node {} (2)
  (0) edge node {} (3)
  (0) edge node {} (4)
  (1) edge node {} (2)
  (1) edge node {} (3)
  (1) edge node {} (4)
  (2) edge node {} (3)
  (2) edge node {} (4)
  (3) edge node {} (4)
  ;

  %% Arbre couvrant
  %\path[-, line width=4pt, cyan]
  %(0) edge node {} (1)
  %(0) edge node {} (2)
  %(1) edge node {} (3)
  %;

  %% Gîtes
  \tikzset{every node/.style={font=\LARGE}}

  \node[yshift=1.2pt] at (0) {\faHome};
  \node[yshift=1.2pt] at (1) {\faHome};
  \node[yshift=  0pt] at (2) {\faHome};
  \node[yshift=  0pt] at (3) {\faHome};
  \node[yshift=  0pt] at (4) {\faHome};
\end{tikzpicture}
\end{figure}

Afin de prouver qu'il y a au minimum $2^n$ chemins possibles, utilisons un sous-graphes $\mathbfcal{H}$ de \G{}. En prenant le sous-graphe $\mathbfcal{H}$ possedant $4$ sommets (gites), il est possible de savoir qu'il y a $2^n$ chemins possibles.
De ce fait, on peut voir, que ce nombre est doublé lorsque le graphe est complet (donc avec $5$ sommets).
En effet, pour chacune des chemins en $n=4$, il y a $2$ chemins possibles lorsque $n=5$.
\begin{figure}[!h]
  \centering
%%%% 5 Pentagrame arbre couvrant
\begin{tikzpicture}[node distance=2cm, baseline=0]
  %$ Sommets
  \tikzset{every state/.style={minimum size=10pt, inner sep=0pt, white}}

  \node[state] at (0, 0.3) (0) {};
  \node[state] at (6, 0.3) (1) {};
  \node[state] at (1,-3)   (2) {};
  \node[state] at (5,-3)   (3) {};
  \node[state] at (3, 2.5) (4) {};

  %% Arêtes
  \path[-, gray]
  (0) edge node {} (1)
  (0) edge node {} (2)
  (0) edge node {} (3)
  (0) edge node {} (4)
  (1) edge node {} (2)
  (1) edge node {} (3)
  (1) edge node {} (4)
  (2) edge node {} (3)
  (2) edge node {} (4)
  (3) edge node {} (4)
  ;

  % Arbre couvrant
  \path[-, line width=4pt, magenta]
  (4) edge node {} (0)
  (0) edge node {} (2)
  (2) edge node {} (3)
  (3) edge node {} (1)
  ;

  %% Gîtes
  \tikzset{every node/.style={font=\LARGE}}

  \node[yshift=1.2pt] at (0) {\faHome};
  \node[yshift=1.2pt] at (1) {\faHome};
  \node[yshift=  0pt] at (2) {\faHome};
  \node[yshift=  0pt] at (3) {\faHome};
  \node[yshift=  0pt] at (4) {\faHome};
\end{tikzpicture}
\begin{tikzpicture}[node distance=2cm, baseline=0]
  %$ Sommets
  \tikzset{every state/.style={minimum size=10pt, inner sep=0pt, white}}

  \node[state] at (0, 0.3) (0) {};
  \node[state] at (6, 0.3) (1) {};
  \node[state] at (1,-3)   (2) {};
  \node[state] at (5,-3)   (3) {};
  \node[state] at (3, 2.5) (4) {};

  %% Arêtes
  \path[-, gray]
  (0) edge node {} (1)
  (0) edge node {} (2)
  (0) edge node {} (3)
  (0) edge node {} (4)
  (1) edge node {} (2)
  (1) edge node {} (3)
  (1) edge node {} (4)
  (2) edge node {} (3)
  (2) edge node {} (4)
  (3) edge node {} (4)
  ;

  % Arbre couvrant
  \path[-, line width=4pt, magenta]
  (0) edge node {} (2)
  (2) edge node {} (3)
  (3) edge node {} (1)
  (1) edge node {} (4)
  ;

  %% Gîtes
  \tikzset{every node/.style={font=\LARGE}}

  \node[yshift=1.2pt] at (0) {\faHome};
  \node[yshift=1.2pt] at (1) {\faHome};
  \node[yshift=  0pt] at (2) {\faHome};
  \node[yshift=  0pt] at (3) {\faHome};
  \node[yshift=  0pt] at (4) {\faHome};
\end{tikzpicture}

\end{figure}

Donc pour $n=5$, le nombre de chemin minimum possible égale $2^4 \cdot 2^1 = 2^5$ chemins possible.
Donc il est possible de dire que pour un graphe complet, il existe au minimum $2^n$ arbre couvrant $n \in \mathbb{N} \geq 4$.


%% At the Moment:
%Une façon compréhensive de le démontrer est de représenter chacun des sommets comme une valeurs binaire.
%Lors d'un compte binaire (\textit{ie}: l'évaluation de $101_2$ qui donne $5$) on réalise qu'il y a jusqu'a $8$ manières différentes d'agencer $3$ valeurs, $3$ sommets.
%Maintenant, si nous retrouvons $4$ somets, il y a alors $4$ valeurs qui peuvent être agencer jusqu'a $16$ façons uniques. 
%Ceci, qui suit cette même relation binaire, gradue en fonction du nombre d'objet à relier par le l'arbre couvrant $\mathbfcal{H}$ du graphe complet, acyclique et non dirigé \G{}.\\
%
%Ainsi, avec $n$ sommets, il y a $2^n$ manières de les agencer afin de couvrir tout les sommets et d'accomplir la tâche requise!


\newpage
\section{c)}

\noindent
\begin{siderules}
  \textbf{Donnez un exemple de points où un arbre couvrant minimal entre ces points est plus long qu'un arbre couvrant minimal avec un nouveau point ajouté (une halte).}
\end{siderules}

\begin{figure}[!h]
  \centering
%%%% 4 BLEU COMME DANS SON DOCUMENT
\begin{tikzpicture}[node distance=2cm, baseline=0]
  %$ Sommets
  \tikzset{every state/.style={minimum size=10pt, inner sep=0pt, white}}

  \node[state]              (0) {};
  \node[state] [right of=0] (1) {};
  \node[state] [below of=0] (2) {};
  \node[state] [below of=1] (3) {};
  \node[text width=4cm, anchor=west, right, align=center] at (4, -1){$3 * 1 = 3 u$};

  %% Arêtes
  \path[-, gray]
  (0) edge node {} (1)
  (0) edge node {} (2)
  (0) edge node {} (3)
  (1) edge node {} (2)
  (1) edge node {} (3)
  (2) edge node {} (3)
  ;

  %% Arbre couvrant
  \path[-, line width=4pt, cyan]
  (0) edge node {} (1)
  (0) edge node {} (2)
  (1) edge node {} (3)
  ;

  %% Gîtes
  \tikzset{every node/.style={font=\LARGE}}

  \node[yshift=1.2pt] at (0) {\faHome};
  \node[yshift=1.2pt] at (1) {\faHome};
  \node[yshift=  0pt] at (2) {\faHome};
  \node[yshift=  0pt] at (3) {\faHome};
\end{tikzpicture}
\end{figure}

Dans cette image, on retrouve un arbre couvrant minimal qui couvre $4$ sommets d'un graph \G{} à $4$ sommets.
Assumant qu'il y a $1$ unité ($u$) de distance entre les coins, du carré formé, adjacants, le graph aurait une longueur suivant l'équation ci-dessus.

\BlankLine

\begin{figure}[!h]
  \centering
%%%% 4 ROUUUUUGE
\begin{tikzpicture}[node distance=2cm]
  %$ Sommets
  \tikzset{every state/.style={minimum size=10pt, inner sep=0pt, white}}

  \node[state]              (0) {};
  \node[state] [right of=0] (1) {};
  \node[state] [below of=0] (2) {};
  \node[state] [below of=1] (3) {};
  \node[text width=4cm, anchor=west, right, align=center] at (4, -1){$4(\frac{1}{2})(1^2 + 1^2)=2^{\frac{1}{2}}=2.8284\dots u$};

  %% Arêtes
  \path[-, gray]
  (0) edge node {} (1)
  (0) edge node {} (2)
  (0) edge node {} (3)
  (1) edge node {} (2)
  (1) edge node {} (3)
  (2) edge node {} (3)
  ;

  %% Arbre couvrant
  \path[-, line width=4pt, red]
  (0) edge node {} (3)
  (1) edge node {} (2)
  ;

  %% Gîtes
  \tikzset{every node/.style={font=\LARGE}}

  \node[yshift=1.2pt] at (0) {\faHome};
  \node[yshift=1.2pt] at (1) {\faHome};
  \node[yshift=  0pt] at (2) {\faHome};
  \node[yshift=  0pt] at (3) {\faHome};
\end{tikzpicture}
\end{figure}

Maintenant, ces gites sont équidistant de cette halte, sont à la même position que le graphe précédent et la distance totale de chacun des gites à la halte va comme suit.

\end{document}